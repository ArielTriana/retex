\section{Introducción}

La Búsqueda y Recuperación de Información (ISR, por sus siglas en inglés: Information Search and Retrieval) es la rama de la ciencia encargada de buscar información en colecciones de documentos digitales. La misma explora los metadatos y el contenido de los documentos para extraer datos que permitan caracterizar la información contenida en los mismos. La ISR no solo se limita a trabajar con documentos de texto, también tiene enfoques de recuperación de información en imágenes, videos, música, entre otros formatos \cite{mri} 

La Recuperación de Información es una rama interdisciplinaria donde generalmente trabajan un grupo de diversos científicos de la bibliotecología, arquitectura de la información, ciencia de la computación, lingüística, inteligencia artifical, archivística, entre otros.

Con la escalada de los grandes volúmenes de datos, surge la necesidad de tener sistemas que recuperen la información deseada por los usuarios, y lo hagan de forma rápida y precisa. De ahí que se dediquen grandes esfuerzos al desarrollo de sistemas que satisfagan estas necesidad.

El proceso de recuperación comienza cuando un usuario hace una consulta de información al sistema. Se entiende por consulta una afirmación formal de la información que necesita el usuario, por ejemplo: ``¿Qué leyes de similitud se deben obedecer al construir modelos aeroelásticos de aeronaves de alta velocidad calentadas?''. Una consulta de forma general no identifica un único documento dentro de la colección de los mismos, si no que identifica un conjunto de estos, donde cada uno de ellos responde a la consulta con un nivel $r$ de relevancia. 

A menudo los sistemas de recuperación desarrollan un ranking según el nivel $r$ de relevancia para una consulta y devuelven al usuario los documentos que contengan mayor $r$.